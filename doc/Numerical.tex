\documentclass[10pt, openany]{book}

\usepackage{fancyhdr}
\usepackage{imakeidx}

\usepackage{amsmath}
\usepackage{amsfonts}

\usepackage{geometry}
\geometry{letterpaper}

\usepackage{fancyvrb}
\usepackage{fancybox}

\usepackage{url}
\usepackage{multicol}
%
% Rules to allow import of graphics files in EPS format
%
\usepackage{graphicx}
\DeclareGraphicsExtensions{.eps}
\DeclareGraphicsRule{.eps}{eps}{.eps}{}
%
%  Include the listings package
%
\usepackage{listings}
%
% Macro definitions
%
\newcommand{\operation}[1]{\textbf{\texttt{#1}}}
\newcommand{\package}[1]{\texttt{#1}}
\newcommand{\function}[1]{\texttt{#1}}
\newcommand{\constant}[1]{\emph{\texttt{#1}}}
\newcommand{\keyword}[1]{\texttt{#1}}
\newcommand{\datatype}[1]{\texttt{#1}}
%
% Front Matter
%
\title{Basic Numerical Analysis Routines}
\author{Brent Seidel \\ Phoenix, AZ}
\date{ \today }
%========================================================
%%% BEGIN DOCUMENT
\begin{document}
%
% Produce the front matter
%
\frontmatter
\maketitle
\begin{center}
This document is \copyright 2024 Brent Seidel.  All rights reserved.

\paragraph{}Note that this is a draft version and not the final version for publication.
\end{center}
\tableofcontents

\mainmatter
%----------------------------------------------------------
\chapter{Introduction}

Back in the 1980s when I was an undergraduate, I took a numerical analysis course and quite enjoyed it.  Then my first job out of college was working on a numerical analysis library for a small startup that went the way of most startups.  I was recently inspired to dig out my old textbook and try implementing some of the routines.  This collection includes some of those, plus others.

Note that some packages are for complex numbers and some are for real numbers.  At some point, they may be combined.  Most packages are generic.  The packages are:

\section{BBS.Numerical.complex}
This is an object oriented collection of complex number routines.  After writing this, I discovered \package{Ada.Numerics.Generic\_Complex\_Types}.  So this package is deprecated in favor of the Ada package.

\section{BBS.Numerical.derivative}
This is a generic package with a real type parameter.  It contains functions to compute the derivative of real valued functions with a single argument.

\section{BBS.Numerical.functions}
This package contains some functions that are used by other packages.

\section{BBS.Numerical.integration\_real}
\section{BBS.Numerical.interpolation}
\section{BBS.Numerical.ode}
\section{BBS.Numerical.polynomial\_complex}
\section{BBS.Numerical.polynomial\_real}
\section{BBS.Numerical.quaternion}
\section{BBS.Numerical.random}
\section{BBS.Numerical.regression}
\section{BBS.Numerical.roots\_complex}
\section{BBS.Numerical.roots\_real}
\section{BBS.Numerical.statistics}
\section{BBS.Numerical.vector}

%----------------------------------------------------------
\chapter{How to Obtain}

This collections is currently available on GitHub at \url{https://github.com/BrentSeidel/Numerical}.

\section{Dependencies}
\subsection{Ada Libraries}
\subsection{Other Libraries}

%----------------------------------------------------------
\chapter{Build Instructions}

%----------------------------------------------------------
\chapter{API Description}
\section{BBS.Numerical.complex}
This package is deprecated in favor of the Ada package \package{Ada.Numerics.Generic\_Complex\_Types}.

\section{BBS.Numerical.derivative}
This is a generic package with a real type parameter, \datatype{F}.

It defines a function type \datatype{test\_func} as \datatype{access function (x : f'Base) return f'Base}.

WARNING:
The calculations here may involve adding small numbers to large numbers and taking the difference of two nearly equal numbers.
The world of computers is not the world of mathematics where numbers have infinite precision.  If you aren't careful, you can get into a situation where $(x + h) = x$, or $f(x) = f(x +/- h)$. For example assume that the float type has 6 digits.  Then, if x is 1,000,000 and h is 1, adding x and h is a wasted operation.

The functions are:
\subsection{Two Point Formulas}
Two point formula.  Use h $>$ 0 for forward-difference and h $<$ 0 for backward-difference.  Derivative calculated at point x.  While this is the basis for defining the derivative in calculus, it generally shouldn't be used.

\function{pt2(f1 : test\_func; x, h : f'Base) return f'Base}
\begin{itemize}
  \item \function{f1} - the function that you with to find the derivative of
  \item \function{x} - the point at which to calculate the derivative
  \item \function{h} - the step size
\end{itemize}

\subsection{Three Point Formulas}
Compute the derivative at $x$ using values at $x$, $x+h$, and $x+2h$.

\function{pt3a(f1 : test\_func; x, h : f'Base) return f'Base}
\begin{itemize}
  \item \function{f1} - the function that you with to find the derivative of
  \item \function{x} - the point at which to calculate the derivative
  \item \function{h} - the step size
\end{itemize}

Compute the derivative at $x$ using values at $x-h$ and $x+h$.

\function{pt3b(f1 : test\_func; x, h : f'Base) return f'Base}
\begin{itemize}
  \item \function{f1} - the function that you with to find the derivative of
  \item \function{x} - the point at which to calculate the derivative
  \item \function{h} - the step size
\end{itemize}

\subsection{Five Point Formulas}

Compute the derivative at $x$ using values at $x-2h$, $x-h$, $x+h$, and $x+2h$.

\function{pt5a(f1 : test\_func; x, h : f'Base) return f'Base}
\begin{itemize}
  \item \function{f1} - the function that you with to find the derivative of
  \item \function{x} - the point at which to calculate the derivative
  \item \function{h} - the step size
\end{itemize}

Compute the derivative at $x$ using values at $x$, $x+h$, $x+2h$, $x+3h$, and $x+4h$.

\function{pt5b(f1 : test\_func; x, h : f'Base) return f'Base}
\begin{itemize}
  \item \function{f1} - the function that you with to find the derivative of
  \item \function{x} - the point at which to calculate the derivative
  \item \function{h} - the step size
\end{itemize}

\section{BBS.Numerical.functions}

\section{BBS.Numerical.integration\_real}

\section{BBS.Numerical.interpolation}

\section{BBS.Numerical.ode}

\section{BBS.Numerical.polynomial\_complex}

\section{BBS.Numerical.polynomial\_real}

\section{BBS.Numerical.quaternion}

\section{BBS.Numerical.random}

\section{BBS.Numerical.regression}

\section{BBS.Numerical.roots\_complex}

\section{BBS.Numerical.roots\_real}

\section{BBS.Numerical.statistics}

\section{BBS.Numerical.vector}


\end{document}

