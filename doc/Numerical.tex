\documentclass[10pt, openany]{book}

\usepackage{fancyhdr}
\usepackage{imakeidx}

\usepackage{amsmath}
\usepackage{amsfonts}

\usepackage{geometry}
\geometry{letterpaper}

\usepackage{fancyvrb}
\usepackage{fancybox}

\usepackage{url}
\usepackage{multicol}
%
% Rules to allow import of graphics files in EPS format
%
\usepackage{graphicx}
\DeclareGraphicsExtensions{.eps}
\DeclareGraphicsRule{.eps}{eps}{.eps}{}
%
%  Include the listings package
%
\usepackage{listings}
%
% Macro definitions
%
\newcommand{\operation}[1]{\textbf{\texttt{#1}}}
\newcommand{\package}[1]{\texttt{#1}}
\newcommand{\function}[1]{\texttt{#1}}
\newcommand{\constant}[1]{\emph{\texttt{#1}}}
\newcommand{\keyword}[1]{\texttt{#1}}
\newcommand{\datatype}[1]{\texttt{#1}}
%
% Front Matter
%
\title{Basic Numerical Analysis Routines}
\author{Brent Seidel \\ Phoenix, AZ}
\date{ \today }
%========================================================
%%% BEGIN DOCUMENT
\begin{document}
%
% Produce the front matter
%
\frontmatter
\maketitle
\begin{center}
This document is \copyright 2024 Brent Seidel.  All rights reserved.

\paragraph{}Note that this is a draft version and not the final version for publication.
\end{center}
\tableofcontents

\mainmatter
%----------------------------------------------------------
\chapter{Introduction}

Back in the 1980s when I was an undergraduate, I took a numerical analysis course and quite enjoyed it.  Then my first job out of college was working on a numerical analysis library for a small startup that went the way of most startups.  I was recently inspired to dig out my old textbook and try implementing some of the routines.  This collection includes some of those, plus others.

Note that some packages are for complex numbers and some are for real numbers.  At some point, they may be combined.  Most packages are generic.  The packages are:

\section{BBS.Numerical.complex}
This is an object oriented collection of complex number routines.  After writing this, I discovered \package{Ada.Numerics.Generic\_Complex\_Types}.  So this package is deprecated in favor of the Ada package.

\section{BBS.Numerical.derivative}
This is a generic package with a real type parameter.  It contains functions to compute the derivative of real valued functions with a single argument.

\section{BBS.Numerical.functions}
This is a generic package with a real type parameter.  It contains some functions that are used by other packages.

\section{BBS.Numerical.integration\_real}
This is a generic package with a real type parameter.  It contains functions to compute integrals of real valued functions with a single argument.

\section{BBS.Numerical.interpolation}
This is a generic package with a real type parameter.  It contains functions to interpolate a value between a set of data points.  The functions can also extrapolate (if using a value of $x$ outside of the range of the data points, however this is likely to be inaccurate.

\section{BBS.Numerical.ode}
This is a generic package with a real type parameter.  It contains functions to solve ordinary differential equations.

\section{BBS.Numerical.polynomial\_complex}
This is a generic package with a complex type parameter (from \package{Ada.Numerics.Generic\_Complex\_Types}.  It contains functions for polynomials.

\section{BBS.Numerical.polynomial\_real}
This is a generic package with a real type parameter.  It contains functions for polynomials.

\section{BBS.Numerical.quaternion}
This is a generic package with a real type parameter.  It contains functions for quaternions.

\section{BBS.Numerical.random}
This is not a generic package.  It contains functions for generating pseudo-random numbers.

\section{BBS.Numerical.regression}
This is a generic package with a real type parameter.  It contains functions for performing regression analysis of data.

\section{BBS.Numerical.roots\_complex}
This is a generic package with a complex type parameter (from \package{Ada.Numerics.Generic\_Complex\_Types}.  It contains functions for finding zeros of functions.

\section{BBS.Numerical.roots\_real}
This is a generic package with a real type parameter.  It contains functions for finding zeros of functions.

\section{BBS.Numerical.statistics}
This is a generic package with a real type parameter.  It contains statistics related functions.

\section{BBS.Numerical.vector}
This is a generic package with a real type parameter.  It contains functions for vectors.


%----------------------------------------------------------
\chapter{How to Obtain}

This collections is currently available on GitHub at \url{https://github.com/BrentSeidel/Numerical}.

\section{Dependencies}
\subsection{Ada Libraries}
The following Ada libraries are used:
\begin{itemize}
  \item \package{Ada.Numerics}
  \item \package{Ada.Numerics.Generic\_Complex\_Types}
  \item \package{Ada.Numerics.Generic\_Complex\_Elementary\_Functions}
  \item \package{Ada.Numerics.Generic\_Elementary\_Functions}
  \item \package{Ada.Text\_IO} (used for debugging purposes)
\end{itemize}
\subsection{Other Libraries}
This library depends on the root package \package{BBS} available at \url{https://github.com/BrentSeidel/BBS-Ada}.
Internal packages used within this library are:
\begin{itemize}
  \item \package{BBS.Numerical.functions}
  \item \package{BBS.Numerical.Integration\_real}
\end{itemize}

%----------------------------------------------------------
\chapter{Usage Instructions}
This is a library of routines intended to be used by some program.  To use these in your program, edit your *\keyword{.gpr} file to include a line to \keyword{with} the path to \keyword{BBS\_Numerical.gpr}.  Then in your Ada code \keyword{with} in the package(s) you need and use the routines.

%----------------------------------------------------------
\chapter{API Description}
\section{BBS.Numerical.complex}
This package is deprecated in favor of the Ada package \package{Ada.Numerics.Generic\_Complex\_Types}.

\section{BBS.Numerical.derivative}
This is a generic package with a real type parameter, \datatype{F}.

It defines a function type \datatype{test\_func} as \datatype{access function (x : f'Base) return f'Base}.

WARNING:
The calculations here may involve adding small numbers to large numbers and taking the difference of two nearly equal numbers.
The world of computers is not the world of mathematics where numbers have infinite precision.  If you aren't careful, you can get into a situation where $(x + h) = x$, or $f(x) = f(x \pm h)$. For example assume that the float type has 6 digits.  Then, if $x$ is $1,000,000$ and $h$ is $1$, adding $x$ and $h$ is a wasted operation.

The functions are:
\subsection{Two Point Formulas}
Two point formula.  Use $h>0$ for forward-difference and $h<0$ for backward-difference.  Derivative calculated at point $x$.  While this is the basis for defining the derivative in calculus, it generally shouldn't be used.

\function{pt2(f1 : test\_func; x, h : f'Base) return f'Base}
\begin{itemize}
  \item \function{f1} - the function that you with to find the derivative of
  \item \function{x} - the point at which to calculate the derivative
  \item \function{h} - the step size
\end{itemize}

\subsection{Three Point Formulas}
Compute the derivative at $x$ using values at $x$, $x+h$, and $x+2h$.

\function{pt3a(f1 : test\_func; x, h : f'Base) return f'Base}
\begin{itemize}
  \item \function{f1} - the function that you with to find the derivative of
  \item \function{x} - the point at which to calculate the derivative
  \item \function{h} - the step size
\end{itemize}

Compute the derivative at $x$ using values at $x-h$ and $x+h$.

\function{pt3b(f1 : test\_func; x, h : f'Base) return f'Base}
\begin{itemize}
  \item \function{f1} - the function that you with to find the derivative of
  \item \function{x} - the point at which to calculate the derivative
  \item \function{h} - the step size
\end{itemize}

\subsection{Five Point Formulas}

Compute the derivative at $x$ using values at $x-2h$, $x-h$, $x+h$, and $x+2h$.

\function{pt5a(f1 : test\_func; x, h : f'Base) return f'Base}
\begin{itemize}
  \item \function{f1} - the function that you with to find the derivative of
  \item \function{x} - the point at which to calculate the derivative
  \item \function{h} - the step size
\end{itemize}

Compute the derivative at $x$ using values at $x$, $x+h$, $x+2h$, $x+3h$, and $x+4h$.

\function{pt5b(f1 : test\_func; x, h : f'Base) return f'Base}
\begin{itemize}
  \item \function{f1} - the function that you with to find the derivative of
  \item \function{x} - the point at which to calculate the derivative
  \item \function{h} - the step size
\end{itemize}

\section{BBS.Numerical.functions}
This is a generic package with a real type parameter, \datatype{F}.  Its primary purpose is to contain functions that are used by other packages in this library.  They may also be useful to the end user.

\subsection{Gamma Function Related}
Some of the probability functions require computing $\Gamma(\frac{n}{2})$.  Since $\Gamma(\frac{1}{2})=\sqrt{pi}$, $\Gamma(1) = 1$, and $\Gamma(a+1) = a*\Gamma(a)$, computing $\Gamma(\frac{n}{2})$, where $n$ is a positive integer is simpler than computing the full $\Gamma$ function.

Computes $\Gamma$ of $2n$.  Note that this will overflow for \datatype{Float} for $n>70$.

\function{gamma2n(n : Positive) return f'Base}
\begin{itemize}
  \item $n$ - A positive integer representing twice the value for computing $\Gamma$.
  \item Returns the actual value.
\end{itemize}

To compute $\Gamma$ of $2n$ where $n>70$, logarithms can be used.  This returns the $log_e(\Gamma(n))$.

\function{lngamma2n(n : Positive) return f'Base}
\begin{itemize}
  \item $n$ - A positive integer representing twice the value for computing $\Gamma$.
  \item Returns the natural log of the value.
\end{itemize}

\subsection{Factorial Related}
The factorial functions are similar to the $\Gamma$ functions.  Since the parameter is not divided by two in this case, overflow will occur for $n>35$.  Thus, as for $\Gamma$, there are two versions.  The first returns the actual value, the second returns the natural log of the value.

\function{factorial(n : Natural) return f'Base}
\begin{itemize}
  \item $n$ - A natural integer representing.
  \item Returns the actual factorial value.
\end{itemize}
\function{lnfact(n : Natural) return f'Base}
\begin{itemize}
  \item $n$ - A natural integer representing.
  \item Returns the natural log of the factorial value.
\end{itemize}

\subsection{Combinatorial Related}
The one function computes the binomial coefficients (aka N choose K).  By definition, this computes:
\begin{displaymath}
  \binom{n}{k} = \frac{n!}{k!(b-k)!}
\end{displaymath}
With some algebra, this becomes:
\begin{displaymath}
  \binom{n}{k} = \prod^k_{i=1}\frac{n+1-i}{i}
\end{displaymath}
 which is easier to compute.  The function is:

\function{nChoosek(n, k : Natural) return f'Base}
\begin{itemize}
  \item $n$ - The number of items.
  \item $k$ - The number of items to choose.
  \item There is a restriction that $n\geq{}k$
\end{itemize}

\section{BBS.Numerical.integration\_real}
This package contains routines to perform numerical integration of a function with one parameter.  Thus it computes an approximation to:
\begin{displaymath}
  y = \int_a^b f(x) dx
\end{displaymath}

\section{BBS.Numerical.interpolation}

\section{BBS.Numerical.ode}

\section{BBS.Numerical.polynomial\_complex}

\section{BBS.Numerical.polynomial\_real}

\section{BBS.Numerical.quaternion}

\section{BBS.Numerical.random}

\section{BBS.Numerical.regression}

\section{BBS.Numerical.roots\_complex}

\section{BBS.Numerical.roots\_real}

\section{BBS.Numerical.statistics}

\section{BBS.Numerical.vector}


\end{document}

